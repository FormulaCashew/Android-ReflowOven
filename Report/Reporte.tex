\documentclass[11pt, a4paper, oneside]{report}

% ==========================================
% PAQUETES Y CONFIGURACIÓN
% ==========================================
\usepackage[utf8]{inputenc}
\usepackage[spanish, es-tabla]{babel}
\usepackage[T1]{fontenc}
\usepackage{helvet} % Fuente más moderna (Helvetica)
\renewcommand{\familydefault}{\sfdefault}

\usepackage{geometry}
\geometry{top=3cm, bottom=3cm, left=3cm, right=2.5cm}

\usepackage{xcolor}
\usepackage{listings}
\usepackage{hyperref}
\usepackage{fancyhdr}
\usepackage{titlesec}
\usepackage{tcolorbox} % Para cajas de texto profesionales
\usepackage{enumitem}
\usepackage{booktabs} % Tablas profesionales
\usepackage{longtable}

% Definición de Colores Corporativos/Técnicos
\definecolor{primaryBlue}{RGB}{0, 51, 102}
\definecolor{accentTeal}{RGB}{0, 128, 128}
\definecolor{lightGray}{RGB}{245, 245, 245}
\definecolor{codeBackground}{RGB}{250, 250, 250}
\definecolor{callStackColor}{RGB}{240, 248, 255}

% Configuración de Hipervínculos
\hypersetup{
    colorlinks=true,
    linkcolor=primaryBlue,
    filecolor=magenta,      
    urlcolor=accentTeal,
    pdftitle={Documentación Técnica - Reflow Oven Controller},
}

% Configuración de Encabezados y Pies de Página
\pagestyle{fancy}
\fancyhf{}
\lhead{\textcolor{gray}{\small Reflow Oven Controller - Android Architecture}}
\rhead{\textcolor{gray}{\small Versión 1.0}}
\cfoot{\thepage}
\renewcommand{\headrulewidth}{0.4pt}
\renewcommand{\footrulewidth}{0.4pt}

% Configuración de Estilos de Código (Kotlin)
\lstdefinestyle{kotlinStyle}{
    language=Java, % Kotlin se basa en JVM, Java es buena aproximación para highlight
    backgroundcolor=\color{codeBackground},   
    commentstyle=\color{gray}\itshape,
    keywordstyle=\color{primaryBlue}\bfseries,
    numberstyle=\tiny\color{gray},
    stringstyle=\color{accentTeal},
    basicstyle=\ttfamily\footnotesize,
    breakatwhitespace=false,         
    breaklines=true,                 
    captionpos=b,                    
    keepspaces=true,                 
    numbers=left,                    
    numbersep=5pt,                  
    showspaces=false,                
    showstringspaces=false,
    showtabs=false,                  
    tabsize=2,
    frame=leftline,
    rulecolor=\color{accentTeal},
    xleftmargin=10pt,
}
\lstset{style=kotlinStyle}

% Formato de Títulos
\titleformat{\chapter}[display]
  {\normalfont\huge\bfseries\color{primaryBlue}}{\chaptertitlename\ \thechapter}{20pt}{\Huge}
\titleformat{\section}
  {\normalfont\Large\bfseries\color{primaryBlue}}{\thesection}{1em}{}
\titleformat{\subsection}
  {\normalfont\large\bfseries\color{accentTeal}}{\thesubsection}{1em}{}

% Definición de Cajas para "Lógica de Llamadas"
\newtcolorbox{calltrace}[1]{
  colback=callStackColor,
  colframe=primaryBlue,
  title=\textbf{Traza de Ejecución: #1},
  fonttitle=\bfseries,
  boxrule=0.5mm,
  sharp corners=south,
  rounded corners=north
}

% ==========================================
% DOCUMENTO
% ==========================================

\begin{document}

% --- PORTADA ---
\begin{titlepage}
    \centering
    \vspace*{2cm}
    
    {\Huge \textbf{\textcolor{primaryBlue}{SISTEMA DE CONTROL PARA HORNO DE REFLUJO}}}\\[0.5cm]
    {\Large \textit{Documentación Técnica de Software Android}}\\[2cm]
    
    \textbf{Arquitectura:} MVVM (Model-View-ViewModel)\\
    \textbf{Lenguaje:} Kotlin / Jetpack Compose\\
    \textbf{Protocolo:} TCP Sockets (Raw ASCII)\\[3cm]
    
    \vfill
    
    \textbf{Departamento de Ingeniería en Automatización}\\
    \today
    
\end{titlepage}

% --- INDICE ---
\tableofcontents
\newpage

% ==========================================
% CAPÍTULO 1
% ==========================================
\chapter{Visión General del Sistema}

\section{Introducción}
El presente documento detalla la arquitectura de software, la dinámica de ejecución y los protocolos de comunicación de la aplicación móvil "Reflow Oven Controller". Este sistema permite la supervisión remota y el control de lazo cerrado de un horno de soldadura mediante una interfaz gráfica reactiva y comunicación de red en tiempo real.

\section{Alcance Técnico}
El software ha sido diseñado bajo los principios de \textit{Clean Architecture} simplificada, implementando el patrón \textbf{MVVM}. Sus responsabilidades principales incluyen:
\begin{enumerate}
    \item \textbf{Gestión de Conectividad:} Establecimiento y mantenimiento de sockets TCP persistentes.
    \item \textbf{Visualización de Datos:} Renderizado de gráficos de temperatura vs. tiempo a 60 FPS.
    \item \textbf{Inyección de Perfiles:} Serialización y transmisión de etapas de soldadura (Soak, Reflow).
    \item \textbf{Concurrencia:} Manejo de hilos en segundo plano para operaciones de E/S.
\end{enumerate}

% ==========================================
% CAPÍTULO 2
% ==========================================
\chapter{Arquitectura de Software}

La aplicación se estructura en tres capas lógicas claramente diferenciadas para garantizar la escalabilidad y testabilidad.

\section{Capa de Presentación (UI Layer)}
Implementada utilizando \textbf{Jetpack Compose}. Esta capa es puramente reactiva; no contiene lógica de negocio, solo lógica de renderizado basada en estados.
\begin{itemize}
    \item \texttt{MainActivity.kt}: Punto de entrada del sistema Android. Configura el tema y el contenedor de superficie.
    \item \texttt{DashboardScreen.kt}: Composable principal. Observa los flujos (\textit{Flows}) del ViewModel y redibuja la interfaz cuando los datos cambian.
    \item \texttt{OvenChart.kt}: Wrapper de interoperabilidad que incrusta una vista de Android clásica (\texttt{MPAndroidChart}) dentro de la jerarquía de Compose.
\end{itemize}

\section{Capa de Lógica de Negocio (ViewModel Layer)}
\texttt{MainViewModel.kt} actúa como el cerebro de la aplicación.
\begin{itemize}
    \item Transforma los datos crudos del repositorio en estados consumibles por la UI (\texttt{StateFlow}).
    \item Gestiona el ciclo de vida de las corrutinas (\texttt{viewModelScope}), asegurando que no haya fugas de memoria si la vista se destruye.
    \item Contiene la lógica de limitación del historial gráfico (buffer circular de 120 puntos).
\end{itemize}

\section{Capa de Datos (Data Layer)}
Responsable de la comunicación con el mundo exterior.
\begin{itemize}
    \item \texttt{ReflowOvenRepository.kt}: Patrón repositorio que abstrae la fuente de datos.
    \item \texttt{WiFiService.kt}: Implementación de bajo nivel de \texttt{CommunicationService}. Maneja los \texttt{Sockets}, \texttt{PrintWriters} y \texttt{Scanners}.
\end{itemize}
